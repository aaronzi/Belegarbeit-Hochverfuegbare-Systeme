\section{Einführung}

Es sollen Fähigkeiten und Fertigkeiten für den Programmentwurf für sicherheitsgerichtete Anlagenmodelle (Funktionale Sicherheit nach DIN EN 61131-6) unter Verwendung von Beschreibungsmitteln und der Programmierung (Normsprachen nach DIN EN 61131-3) am Beispiel eines Silos mit Fördereinrichtung aufgebaut werden. Hierzu sollen zunächst unter Verwendung der textbasierten Programmiersprache \glqq Strukturierter Text, ST\grqq{} sicherheitsgerichtete Programmelemente entwickelt werden. Für diesen Zweck wird die Siemens S7-1500 Industriesteuerung inklusive der dezentralen Peripherie ET 200 SP und deren Programmierumgebung TIA Portal V17 verwendet.

\subsection{Voraussetzungen}

Um die nachfolgend beschriebene Anlage in Betrieb nehmen und Fehler simulieren zu können, wird ein Bachelor-Abschluss in Elektrotechnik oder in einem anderen ingenieurwissenschaftlichen Studiengang vorausgesetzt. Zusätzlich wird das Wissen aus den Vorlesungen der Bachelor-Module \glqq Grundlagen der Automation\grqq{}, \glqq Prozesssteuerungssysteme\grqq{} und \glqq Projekt: Prozesssteuerungssysteme\grqq{} und der Nachweis der erfolgreichen Teilnahme an den jeweiligen Laborpraktika verlangt. Durch die erfolgreiche Teilnahme weist der Studierende die notwendigen Fähigkeiten im Bereich der ST-Programmierung nach.